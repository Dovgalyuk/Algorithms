\documentclass[10pt,twoside,openany]{book}

\usepackage[
  a5paper, mag=1000,
  left=2cm, right=2cm, top=2cm, bottom=2cm, headsep=0.7cm, footskip=1.27cm
]{geometry}

%\pdfpagewidth 145mm
%\pdfpageheight 215mm

%\textwidth=120mm
%\textheight=195mm

\usepackage[T2A]{fontenc}
\usepackage[utf8]{inputenc}
%\usepackage{ucs}
\usepackage[english,russian]{babel}
\usepackage{textcomp}
%\usepackage{cmap}
%\IfFileExists{pscyr.sty}{\usepackage{pscyr}}{} % Нормальные шрифты
%\usepackage{amsmath}
%\usepackage{tabularx}
%\usepackage{array}
\usepackage{multirow}
%\usepackage{amssymb}

\usepackage{graphicx}

\usepackage{tabularx}

% Короткие заголовки только в колонтитулах
\usepackage[toctitles]{titlesec}

%
%% цвет в документе
%\usepackage[usenames]{color}
%\usepackage{colortbl}
%
%% собственные разделы
%\usepackage{titletoc}
%
%% Форматирование чисел
%\usepackage{siunitx}
%\sisetup{range-phrase = \ldots}
%
%% оглавление + ссылки
%\usepackage[pdftex]{hyperref}
%
%\hypersetup{%
%    pdfborder = {0 0 0},
%    colorlinks,
%    citecolor=red,
%    filecolor=Darkgreen,
%    linkcolor=blue,
%    urlcolor=blue
%}
%
%\usepackage{hypcap}
%\usepackage[numbered]{bookmark} % для цифр в закладках в pdf viewer'ов
%
%\usepackage{tocloft}
%
%\renewcommand{\cftsecaftersnum}{.}
%\renewcommand{\cftsubsecaftersnum}{.}
%
%\setcounter{tocdepth}{4}
%\setcounter{secnumdepth}{4}
%
%\renewcommand\cftsecleader{\cftdotfill{\cftdotsep}}
%\renewcommand{\cfttoctitlefont}{\hfill\Large\bfseries}
%\renewcommand{\cftaftertoctitle}{\hfill}
%
%% листинги
\usepackage{listings}
\usepackage{color}
%\usepackage{caption} % для подписей под листингами и таблиц
%%\usepackage[scaled]{beramono}
%
%% графика
%\usepackage[pdftex]{graphicx}
%\graphicspath{{./fig/}} % папка с изображениями
%%\usepackage[tikz]{bclogo}
%%\usepackage{wrapfig}
%%\usepackage{tikz}
%%\usetikzlibrary{shapes.geometric,backgrounds,calc,positioning,arrows,
%%                chains,decorations.markings,patterns,fadings,shapes.multipart,trees}
%
%%\usepackage{setspace} % для полуторного интервала
%%\onehalfspacing % сам полуторный интервал
%
\usepackage{indentfirst} % отступ в первом абзаце
%
%% точки в секция и подсекциях
%% \usepackage{secdot}
%% \sectiondot{subsection}

% Подавление висячих строк
\clubpenalty=10000
\widowpenalty=10000

%% колонтитулы
%%\usepackage{fancybox,fancyhdr}
%%\pagestyle{fancy}
%%\fancyhf{}
%%\fancyhead[C]{\small{C++ and Qt}} % шапка верхнего колонтитула!!!
%%\fancyfoot[C]{\small{\thepage}} % шапка нижнего колонтитула!!!
%
%% стили листингов кодов

\definecolor{commentGreen}{rgb}{0,0.6,0}
\lstdefinestyle{CPlusPlus}{
  language=C++,
  basicstyle=\small\ttfamily,
  breakatwhitespace=true,
  breaklines=true,
  showstringspaces=false,
  keywordstyle=\color{blue}\ttfamily,
  stringstyle=\color{red}\ttfamily,
  commentstyle=\color{commentGreen}\ttfamily,
  morecomment=[l][\color{magenta}]{\#},
  numbers=left,
  xleftmargin=1cm
}
\lstset{style=CPlusPlus}
\lstset{inputencoding=utf8, extendedchars=\true}
\lstset{escapeinside={(*@}{@*)}}

\addto\captionsrussian{\renewcommand\chaptername{Раздел}}
\newcommand{\chaptershort}[1]{
   \markboth{\MakeUppercase{\chaptername{ }\thechapter.\hspace{1em}#1}}{}
}


\begin{document}

\begin{titlepage}
\newpage

\begin{center}
Министерство образования и науки Российской Федерации \\
\vspace{0.5em}
Федеральное государственное бюджетное образовательное учреждение
высшего профессионального образования
<<Новгородский государственный университет \\ имени Ярослава Мудрого>> \\*
\hrulefill
\end{center}
 
\vspace{4em}

\begin{center}
\bfseries 
\fontsize{15pt}{5pt}\selectfont 
П. М. Довгалюк
\end{center}

\vspace{0.7em}

\begin{center}
\scshape\bfseries\fontsize{20pt}{20pt}\selectfont 
Практикум на языке C++
\end{center}

% \vspace{0.5em}
% \begin{center}
% \scshape\large
% Монография
% \end{center}
 
\vspace{\fill}

\begin{center}
Великий Новгород \\ 2021
\end{center}

\end{titlepage}

%%%%%%%%%%%%%%%%%%%%%%%%%%%%%%%%%%%%%%%%%%%%%%%%%%%%%%%%%%%%%%%%%%%%%%%%%%%%% 
%% Используйте этот файл как шаблон для выходных сведений о вашем издании  %%
%% на обороте титульного листа                                             %%
%%%%%%%%%%%%%%%%%%%%%%%%%%%%%%%%%%%%%%%%%%%%%%%%%%%%%%%%%%%%%%%%%%%%%%%%%%%%%
{%\large
\parindent=0cm
\thispagestyle{empty}

\begin{tabularx}{\textwidth}{cp{0.4\textwidth}p{0.6\textwidth}}
ББК & 32.973.26-018.2 & Печатается по решению \\
 & Д58 &  РИС НовГУ \\
\end{tabularx}

\vspace{1cm}

\begin{center}
Рецензенты: \\
рецензент 1 \\
рецензент 2 \\
\end{center}

\vspace{1cm}

% \vspace{0.06cm}
% \vspace{0.6cm} 

\begin{tabularx}{\textwidth}{cp{0.9\textwidth}}
& {\bf Довгалюк, П. М.} \\
Д58 & \hspace{3ex} Практикум на языке C++ / П.~М.~Довгалюк;
НовГУ им.~Ярослава Мудрого. -- Великий Новгород, 2021. -- xx с. \vspace{0.32cm} \\
& \hspace{3ex} ...
\\
& \hspace{3ex} Пособие предназначено для студентов первых курсов технических специальностей. \\
\end{tabularx}
\\
\begin{flushright}
%УДК 51(075.8) + 51(075.8) \\
ББК~32.973.26-018.2 \\
\end{flushright}

\vfill
\begin{tabularx}{\textwidth}{p{5cm}ll}
& \copyright & Новгородский государственный \\ && университет,~2021 \\ 
& \copyright & П. М. Довгалюк,~2021 \\ 
\end{tabularx}
}


\tableofcontents

\clearpage

\chapter{std::vector}

\section{new[]/delete[] vs std::vector}

Перепишите следующую программу, используя {\tt std::vector}.

Последовательно избавьтесь от следующих конструкций:
\begin{itemize}
    \item Операторы {\tt new[]/delete[]} (переменные {\tt a} и {\tt b}
          превратите в объекты {\tt std::vector}.
    \item Переменная {\tt m} и цикл для вычисления её значения.
\end{itemize}

\lstinputlisting{sources/vector1.cpp}

\subsection*{Контрольные вопросы}

\begin{itemize}
    \item Что делает исходная программа?
    \item Какие ограничения неявно наложены на значение переменной {\tt n}?
    \item На сколько строк удалось сократить исходный код, используя {\tt std::vector}?
    \item Когда освобождается память, используемая {\tt std::vector}?
\end{itemize}

\chapter{std::array}

\chapter{std::list}

\chapter{Строки}

\section{Изограммы}

An isogram (also known as a "nonpattern word") is a word or phrase without a repeating letter, however spaces and hyphens are allowed to appear multiple times.

\section{Боб}

Bob is a lackadaisical teenager. In conversation, his responses are very limited.

Bob answers 'Sure.' if you ask him a question, such as "How are you?".

He answers 'Whoa, chill out!' if you YELL AT HIM (in all capitals).

He answers 'Calm down, I know what I'm doing!' if you yell a question at him.

He says 'Fine. Be that way!' if you address him without actually saying anything.

He answers 'Whatever.' to anything else.

Bob's conversational partner is a purist when it comes to written communication and always follows normal rules regarding sentence punctuation in English.

\section{Crypto square}

Implement the classic method for composing secret messages called a square code.

Given an English text, output the encoded version of that text.

First, the input is normalized: the spaces and punctuation are removed from the English text and the message is downcased.

Then, the normalized characters are broken into rows. These rows can be regarded as forming a rectangle when printed with intervening newlines.

For example, the sentence

"If man was meant to stay on the ground, god would have given us roots."
is normalized to:

"ifmanwasmeanttostayonthegroundgodwouldhavegivenusroots"
The plaintext should be organized in to a rectangle. The size of the rectangle (r x c) should be decided by the length of the message, such that c >= r and c - r <= 1, where c is the number of columns and r is the number of rows.

Our normalized text is 54 characters long, dictating a rectangle with c = 8 and r = 7:

"ifmanwas"
"meanttos"
"tayonthe"
"groundgo"
"dwouldha"
"vegivenu"
"sroots  "

The coded message is obtained by reading down the columns going left to right.

The message above is coded as:

"imtgdvsfearwermayoogoanouuiontnnlvtwttddesaohghnsseoau"
Output the encoded text in chunks that fill perfect rectangles (r X c), with c chunks of r length, separated by spaces. For phrases that are n characters short of the perfect rectangle, pad each of the last n chunks with a single trailing space.

"imtgdvs fearwer mayoogo anouuio ntnnlvt wttddes aohghn  sseoau "
Notice that were we to stack these, we could visually decode the ciphertext back in to the original message:

"imtgdvs"
"fearwer"
"mayoogo"
"anouuio"
"ntnnlvt"
"wttddes"
"aohghn "
"sseoau "

\section{Подстроки}

Given a string of digits, output all the contiguous substrings of length n in that string in the order that they appear.

For example, the string "49142" has the following 3-digit series:

"491"
"914"
"142"
And the following 4-digit series:

"4914"
"9142"
And if you ask for a 6-digit series from a 5-digit string, you deserve whatever you get.

Note that these series are only required to occupy adjacent positions in the input; the digits need not be numerically consecutive.



\chapter{Алгоритмы}

\section{Двоичный поиск}

\chapter{Структуры данных}

\section{Кольцевой буфер}

\chapter{Хеширование}

\section{Анаграммы}

An anagram is a rearrangement of letters to form a new word.
Given a word and a list of candidates, select the sublist of anagrams of the given word.

Given "listen" and a list of candidates like "enlists" "google" "inlets" "banana"
the program should return a list containing "inlets".

\section{Анаграммы 2}

Найти все анаграммы из слов-кандидатов для всех подмножеств букв из исходного слова.

\chapter{std::map}

\section{Подсчёт слов}

Given a phrase, count the occurrences of each word in that phrase.

For the purposes of this exercise you can expect that a word will always be one of:

A number composed of one or more ASCII digits (ie "0" or "1234") OR
A simple word composed of one or more ASCII letters (ie "a" or "they") OR
A contraction of two simple words joined by a single apostrophe (ie "it's" or "they're")
When counting words you can assume the following rules:

The count is case insensitive (ie "You", "you", and "YOU" are 3 uses of the same word)
The count is unordered; the tests will ignore how words and counts are ordered
Other than the apostrophe in a contraction all forms of punctuation are ignored
The words can be separated by any form of whitespace (ie "\textbackslash t", "\textbackslash n", " ")


\end{document}
