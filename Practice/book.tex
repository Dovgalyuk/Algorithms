\documentclass[10pt,twoside,openany]{book}

\usepackage[
  a5paper, mag=1000,
  left=2cm, right=2cm, top=2cm, bottom=2cm, headsep=0.7cm, footskip=1.27cm
]{geometry}

%\pdfpagewidth 145mm
%\pdfpageheight 215mm

%\textwidth=120mm
%\textheight=195mm

\usepackage[T2A]{fontenc}
\usepackage[utf8]{inputenc}
%\usepackage{ucs}
\usepackage[english,russian]{babel}
\usepackage{textcomp}
%\usepackage{cmap}
%\IfFileExists{pscyr.sty}{\usepackage{pscyr}}{} % Нормальные шрифты
%\usepackage{amsmath}
%\usepackage{tabularx}
%\usepackage{array}
\usepackage{multirow}
%\usepackage{amssymb}

\usepackage{graphicx}

\usepackage{tabularx}

% Короткие заголовки только в колонтитулах
\usepackage[toctitles]{titlesec}

%
%% цвет в документе
%\usepackage[usenames]{color}
%\usepackage{colortbl}
%
%% собственные разделы
%\usepackage{titletoc}
%
%% Форматирование чисел
%\usepackage{siunitx}
%\sisetup{range-phrase = \ldots}
%
%% оглавление + ссылки
%\usepackage[pdftex]{hyperref}
%
%\hypersetup{%
%    pdfborder = {0 0 0},
%    colorlinks,
%    citecolor=red,
%    filecolor=Darkgreen,
%    linkcolor=blue,
%    urlcolor=blue
%}
%
%\usepackage{hypcap}
%\usepackage[numbered]{bookmark} % для цифр в закладках в pdf viewer'ов
%
%\usepackage{tocloft}
%
%\renewcommand{\cftsecaftersnum}{.}
%\renewcommand{\cftsubsecaftersnum}{.}
%
%\setcounter{tocdepth}{4}
%\setcounter{secnumdepth}{4}
%
%\renewcommand\cftsecleader{\cftdotfill{\cftdotsep}}
%\renewcommand{\cfttoctitlefont}{\hfill\Large\bfseries}
%\renewcommand{\cftaftertoctitle}{\hfill}
%
%% листинги
\usepackage{listings}
\usepackage{color}
%\usepackage{caption} % для подписей под листингами и таблиц
%%\usepackage[scaled]{beramono}
%
%% графика
%\usepackage[pdftex]{graphicx}
%\graphicspath{{./fig/}} % папка с изображениями
%%\usepackage[tikz]{bclogo}
%%\usepackage{wrapfig}
%%\usepackage{tikz}
%%\usetikzlibrary{shapes.geometric,backgrounds,calc,positioning,arrows,
%%                chains,decorations.markings,patterns,fadings,shapes.multipart,trees}
%
%%\usepackage{setspace} % для полуторного интервала
%%\onehalfspacing % сам полуторный интервал
%
\usepackage{indentfirst} % отступ в первом абзаце
%
%% точки в секция и подсекциях
%% \usepackage{secdot}
%% \sectiondot{subsection}

% Подавление висячих строк
\clubpenalty=10000
\widowpenalty=10000

%% колонтитулы
%%\usepackage{fancybox,fancyhdr}
%%\pagestyle{fancy}
%%\fancyhf{}
%%\fancyhead[C]{\small{C++ and Qt}} % шапка верхнего колонтитула!!!
%%\fancyfoot[C]{\small{\thepage}} % шапка нижнего колонтитула!!!
%
%% стили листингов кодов

\definecolor{commentGreen}{rgb}{0,0.6,0}
\lstdefinestyle{CPlusPlus}{
  language=C++,
  basicstyle=\small\ttfamily,
  breakatwhitespace=true,
  breaklines=true,
  showstringspaces=false,
  keywordstyle=\color{blue}\ttfamily,
  stringstyle=\color{red}\ttfamily,
  commentstyle=\color{commentGreen}\ttfamily,
  morecomment=[l][\color{magenta}]{\#},
  numbers=left,
  xleftmargin=1cm
}
\lstset{style=CPlusPlus}
\lstset{inputencoding=utf8, extendedchars=\true}
\lstset{escapeinside={(*@}{@*)}}

\addto\captionsrussian{\renewcommand\chaptername{Раздел}}
\newcommand{\chaptershort}[1]{
   \markboth{\MakeUppercase{\chaptername{ }\thechapter.\hspace{1em}#1}}{}
}


\begin{document}

\begin{titlepage}
\newpage

\begin{center}
Министерство образования и науки Российской Федерации \\
\vspace{0.5em}
Федеральное государственное бюджетное образовательное учреждение
высшего профессионального образования
<<Новгородский государственный университет \\ имени Ярослава Мудрого>> \\*
\hrulefill
\end{center}
 
\vspace{4em}

\begin{center}
\bfseries 
\fontsize{15pt}{5pt}\selectfont 
П. М. Довгалюк
\end{center}

\vspace{0.7em}

\begin{center}
\scshape\bfseries\fontsize{20pt}{20pt}\selectfont 
Практикум на языке C++
\end{center}

% \vspace{0.5em}
% \begin{center}
% \scshape\large
% Монография
% \end{center}
 
\vspace{\fill}

\begin{center}
Великий Новгород \\ 2021
\end{center}

\end{titlepage}

%%%%%%%%%%%%%%%%%%%%%%%%%%%%%%%%%%%%%%%%%%%%%%%%%%%%%%%%%%%%%%%%%%%%%%%%%%%%% 
%% Используйте этот файл как шаблон для выходных сведений о вашем издании  %%
%% на обороте титульного листа                                             %%
%%%%%%%%%%%%%%%%%%%%%%%%%%%%%%%%%%%%%%%%%%%%%%%%%%%%%%%%%%%%%%%%%%%%%%%%%%%%%
{%\large
\parindent=0cm
\thispagestyle{empty}

\begin{tabularx}{\textwidth}{cp{0.4\textwidth}p{0.6\textwidth}}
ББК & 32.973.26-018.2 & Печатается по решению \\
 & Д58 &  РИС НовГУ \\
\end{tabularx}

\vspace{1cm}

\begin{center}
Рецензенты: \\
рецензент 1 \\
рецензент 2 \\
\end{center}

\vspace{1cm}

% \vspace{0.06cm}
% \vspace{0.6cm} 

\begin{tabularx}{\textwidth}{cp{0.9\textwidth}}
& {\bf Довгалюк, П. М.} \\
Д58 & \hspace{3ex} Практикум на языке C++ / П.~М.~Довгалюк;
НовГУ им.~Ярослава Мудрого. -- Великий Новгород, 2021. -- xx с. \vspace{0.32cm} \\
& \hspace{3ex} ...
\\
& \hspace{3ex} Пособие предназначено для студентов первых курсов технических специальностей. \\
\end{tabularx}
\\
\begin{flushright}
%УДК 51(075.8) + 51(075.8) \\
ББК~32.973.26-018.2 \\
\end{flushright}

\vfill
\begin{tabularx}{\textwidth}{p{5cm}ll}
& \copyright & Новгородский государственный \\ && университет,~2021 \\ 
& \copyright & П. М. Довгалюк,~2021 \\ 
\end{tabularx}
}


\tableofcontents

\clearpage

\chapter{std::vector}

\section{new[]/delete[] vs std::vector}

Перепишите следующую программу, используя {\tt std::vector}.

Последовательно избавьтесь от следующих конструкций:
\begin{itemize}
    \item Операторы {\tt new[]/delete[]} (переменные {\tt a} и {\tt b}
          превратите в объекты {\tt std::vector}.
    \item Переменная {\tt m} и цикл для вычисления её значения.
\end{itemize}

\lstinputlisting{sources/vector1.cpp}

\subsection*{Контрольные вопросы}

\begin{itemize}
    \item Что делает исходная программа?
    \item Какие ограничения неявно наложены на значение переменной {\tt n}?
    \item На сколько строк удалось сократить исходный код, используя {\tt std::vector}?
    \item Когда освобождается память, используемая {\tt std::vector}?
\end{itemize}

\chapter{std::array}

\section{Массивы-значения}
\label{array-value}

Сравните две программы ниже. Предположите, какие значения выводятся в каждой из них,
а затем попробуйте запустить их и проверить результат.

\lstinputlisting{sources/array1.cpp}

\lstinputlisting{sources/array2.cpp}

\subsection*{Контрольные вопросы}

\begin{itemize}
    \item Что выведут первая и вторая программы?
    \item Как можно было бы доработать вторую, чтобы получался такой же результат, как и в первой?
    \item Какое преимущество даёт использование {\tt std::array}?
    \item Что нужно изменить, чтобы не переделывать функцию {\tt sum} при изменении размера массива?
\end{itemize}

\chapter{std::list}

\section{Список вместо массива}

Проанализируйте следующую программу. Попробуйте выяснить, как время её работы растёт
с увеличением входного параметра $n$.

\lstinputlisting{sources/list1.cpp}

Перепишите программу с использованием контейнера {\tt std::list} вместо массива.
Для вставки значений в список используйте функцию {\tt std::list::insert}.

\subsection*{Контрольные вопросы}

\begin{itemize}
    \item В чём заключается ошибка программиста, который написал исходную программу?
    \item Какими способами можно просмотреть все элементы списка, чтобы вывести их на экран?
\end{itemize}

\chapter{Строки}

\section{Изограммы}

Изограмма --- это слово (или фраза), в котором не повторяются буквы.

Напишите программу, которая проверяет, является ли введённая строка изограммой.
Строка состоит из множества латинских букв в разном регистре, пробелов и знаков препинания.
Причём в изограмме пробелы или знаки препинания могут повторяться.

При реализации используйте следующие конструкции:
\begin{itemize}
    \item Функцию {\tt std::isalpha}
    \item Функцию {\tt std::tolower} или {\tt std::toupper}
\end{itemize}

\subsection*{Контрольные вопросы}

В вопросах подразумевается, что для входных данных используется кодировка ASCII.

\begin{itemize}
    \item Какой фрагмент кода мог бы заменить функцию {\tt std::isalpha}?
    \item Как можно реализовать функцию {\tt std::tolower}?
\end{itemize}

\section{Боб}

Боб не очень любит разговаривать, поэтому использует небольшой набор реплик.

Он отвечает <<Sure.>> на любой вопрос, например <<How are you?>>.

Он говорит <<Whoa, chill out!>>, если вы КРИЧИТЕ НА НЕГО (то есть используете одни заглавные буквы).

Он отвечает <<Calm down, I know what I'm doing!>> на вопрос, в котором вы кричите.

Он говорит <<Fine. Be that way!>> если вы обращаетесь к нему, но ничего
не говорите (это значит, что входная строка программы состоит из пробельных символов).

Он говорит <<Whatever.>> во всех остальных случаях.

Напишите диалоговую программу, которая имитирует разговор с Бобом. На каждую входную строку она
должна выводить такой ответ, какой давал бы Боб в такой же ситуации. Входные реплики для программы
должны подчиняться правилам пунктуации английского языка.

\subsection*{Контрольные вопросы}

\begin{itemize}
    \item Каким способом проверяется, что все буквы во введённой строке заглавные?
    \item Какую функцию стандартной библиотеки можно использовать для проверки пробельных символов?
\end{itemize}

\section{Crypto square}

Implement the classic method for composing secret messages called a square code.

Given an English text, output the encoded version of that text.

First, the input is normalized: the spaces and punctuation are removed from the English text and the message is downcased.

Then, the normalized characters are broken into rows. These rows can be regarded as forming a rectangle when printed with intervening newlines.

For example, the sentence

"If man was meant to stay on the ground, god would have given us roots."
is normalized to:

"ifmanwasmeanttostayonthegroundgodwouldhavegivenusroots"
The plaintext should be organized in to a rectangle. The size of the rectangle (r x c) should be decided by the length of the message, such that c >= r and c - r <= 1, where c is the number of columns and r is the number of rows.

Our normalized text is 54 characters long, dictating a rectangle with c = 8 and r = 7:

"ifmanwas"
"meanttos"
"tayonthe"
"groundgo"
"dwouldha"
"vegivenu"
"sroots  "

The coded message is obtained by reading down the columns going left to right.

The message above is coded as:

"imtgdvsfearwermayoogoanouuiontnnlvtwttddesaohghnsseoau"
Output the encoded text in chunks that fill perfect rectangles (r X c), with c chunks of r length, separated by spaces. For phrases that are n characters short of the perfect rectangle, pad each of the last n chunks with a single trailing space.

"imtgdvs fearwer mayoogo anouuio ntnnlvt wttddes aohghn  sseoau "
Notice that were we to stack these, we could visually decode the ciphertext back in to the original message:

"imtgdvs"
"fearwer"
"mayoogo"
"anouuio"
"ntnnlvt"
"wttddes"
"aohghn "
"sseoau "

\subsection*{Контрольные вопросы}

\begin{itemize}
    \item
\end{itemize}

\section{Подстроки}

На входе в программу поступает строка из цифр и число $n$. Программа должна вывести все
непрерывные подстроки исходной строки длины $n$.

Например, для строки $49142$ и $n=3$ вывод будет таким: $491$, $914$, $142$.

А для $n=4$ таким: $4914$, $9142$.

\subsection*{Контрольные вопросы}

\begin{itemize}
    \item Какую стандартную функцию можно использовать для получения подстрок?
\end{itemize}

\chapter{Алгоритмы}

\section{Сортировка}

\subsection*{Контрольные вопросы}

\begin{itemize}
    \item
\end{itemize}

\section{Двоичный поиск}

\subsection*{Контрольные вопросы}

\begin{itemize}
    \item
\end{itemize}

\section{std::accumulate}

Переделайте функции {\tt sum} в программах из задания~\ref{array-value},
используя алгоритм {\tt std::accumulate}.
Для этого потребуется подключить заголовочный файл {\tt numeric}.

\subsection*{Контрольные вопросы}

\begin{itemize}
    \item Как переделать вызов {\tt std::accumulate}, чтобы вычислялась сумма элементов
          из первой половины массива?
\end{itemize}

\section{Снова std::accumulate}

Переделайте программы из предыдущего задания, чтобы вместо суммы функция {\tt std::accumulate} считала
произведение элементов.

\subsection*{Контрольные вопросы}

\begin{itemize}
    \item Какое значение должно получиться, если в массиве не будет ни одного элемента?
\end{itemize}

\chapter{Структуры данных}

\section{Кольцевой буфер}

\subsection*{Контрольные вопросы}

\begin{itemize}
    \item
\end{itemize}

\chapter{Хеширование}

\section{Анаграммы}

Анаграмма --- это слово, полученное из исходного перестановкой букв.
Напишите программу, которая определяет, какие слова из списка являются анаграммами
заданного слова.

Например, для слова <<listen>> и списка <<enlists>>, <<google>>, <<inlets>>, <<banana>>
программа должна вывести список из одного элемента: <<inlets>>.

Предварительный отсев слов-кандидатов выполняйте с помощью вычисления
хеш-функции от этих строк.

\subsection*{Контрольные вопросы}

\begin{itemize}
    \item Какую хеш-функцию вы выбрали?
    \item Можно ли для тех же целей использовать стандартную
          реализацию {\tt std::hash<std::string>}?
    \item Что кроме хеш-функции нужно использовать, чтобы убедиться, что буквы в строках совпадают?
\end{itemize}

\section{Коллизии}

Найти все анаграммы из слов-кандидатов для всех подмножеств букв из исходного слова.

Получить подмножества букв можно несколькими способами:
\begin{itemize}
    \item Рекурсивный перебор. На каждом шаге обрабатывается одна буква входного слова.
          Для каждого варианта (когда буква включается в результирующее слово и когда не включается)
          рекурсивный поиск продолжается со следующей буквой.
    \item Последовательно увеличивающиеся целые числа можно раскладывать на двоичные разряды.
          Каждый разряд сопоставляется одной букве. Если разряд $1$, то буква включается
          в результирующее подмножество.
\end{itemize}

Для хранения списка слов-кандидатов используйте контейнер {\tt std::unordered\_set}.
Этот контейнер использует хеширование для поиска и упорядочивания элементов.
Используйте второй параметр шаблона контейнера для указания собственной хеш-функции,
чтобы сравнивать строки без учёта порядка букв.
После генерации очередного подмножества букв можно проверять, если ли уже слово с такими
буквами в контейнере.

Проанализируйте, как часто возникают коллизии, если использовать только хеширование,
без сложной функции сравнения строк.

\subsection*{Контрольные вопросы}

\begin{itemize}
    \item С чем связано, что строки из разных букв могут получать одни и те же значения
          хеш-функции?
    \item Приведите пример двух строк из разных букв, но с одинаковым значением хеш-функции.
\end{itemize}

\chapter{std::map}

\section{Поиск}

Перепишите следующую программу, используя {\tt std::map}.

Выполните следующие действия:
\begin{itemize}
    \item Замените массивы {\tt word} и {\tt def} на переменную-контейнер {\tt std::map}.
    \item Замените алгоритм поиска на функцию {\tt find} контейнера {\tt std::map}.
\end{itemize}

\lstinputlisting{sources/map1.cpp}

\subsection*{Контрольные вопросы}

\begin{itemize}
    \item Что делает исходная программа?
    \item Какой тип ключа применяется в использованном контейнере {\tt std::map}?
    \item Что возвращает функция {\tt find}, если нужный элемент не найден?
\end{itemize}

\section{Подсчёт слов}

Напишите программу, которая выводит сколько раз каждое слово из входной строки повторяется в ней.

Слова могут состоять из цифр и латинских букв. Слова с разным регистром букв считаются одинаковыми.
Знаки препинания, пробелы, переносы строк и символы табуляции нужно игнорировать.

\subsection*{Подзадачи}

\begin{itemize}
    \item Используйте {\tt std::map<std::string, int>} для хранения информации о количестве повторов.
    \item Попробуйте не считывать весь текст целиком, а читать по одному символу из входного потока.
    \item Напишите два варианта программы: с преобразованием хранимых в {\tt std::map} слов и без него.
          Для второго варианта замените функцию сравнения (третий параметр шаблона) на собственную.
\end{itemize}

\subsection*{Контрольные вопросы}

\begin{itemize}
    \item Определён ли порядок элементов в {\tt std::map}? Можно ли его изменить?
\end{itemize}

\end{document}
