\usepackage[
  a5paper, mag=1000,
  left=2cm, right=2cm, top=2cm, bottom=2cm, headsep=0.7cm, footskip=1.27cm
]{geometry}

%\pdfpagewidth 145mm
%\pdfpageheight 215mm

%\textwidth=120mm
%\textheight=195mm

\usepackage[T2A]{fontenc}
\usepackage[utf8]{inputenc}
%\usepackage{ucs}
\usepackage[english,russian]{babel}
\usepackage{textcomp}
%\usepackage{cmap}
%\IfFileExists{pscyr.sty}{\usepackage{pscyr}}{} % Нормальные шрифты
%\usepackage{amsmath}
%\usepackage{tabularx}
%\usepackage{array}
\usepackage{multirow}
%\usepackage{amssymb}

\usepackage{graphicx}

\usepackage{tabularx}

% Короткие заголовки только в колонтитулах
\usepackage[toctitles]{titlesec}

%
%% цвет в документе
%\usepackage[usenames]{color}
%\usepackage{colortbl}
%
%% собственные разделы
%\usepackage{titletoc}
%
%% Форматирование чисел
%\usepackage{siunitx}
%\sisetup{range-phrase = \ldots}
%
%% оглавление + ссылки
%\usepackage[pdftex]{hyperref}
%
%\hypersetup{%
%    pdfborder = {0 0 0},
%    colorlinks,
%    citecolor=red,
%    filecolor=Darkgreen,
%    linkcolor=blue,
%    urlcolor=blue
%}
%
%\usepackage{hypcap}
%\usepackage[numbered]{bookmark} % для цифр в закладках в pdf viewer'ов
%
%\usepackage{tocloft}
%
%\renewcommand{\cftsecaftersnum}{.}
%\renewcommand{\cftsubsecaftersnum}{.}
%
%\setcounter{tocdepth}{4}
%\setcounter{secnumdepth}{4}
%
%\renewcommand\cftsecleader{\cftdotfill{\cftdotsep}}
%\renewcommand{\cfttoctitlefont}{\hfill\Large\bfseries}
%\renewcommand{\cftaftertoctitle}{\hfill}
%
%% листинги
\usepackage{listings}
\usepackage{color}
%\usepackage{caption} % для подписей под листингами и таблиц
%%\usepackage[scaled]{beramono}
%
%% графика
%\usepackage[pdftex]{graphicx}
%\graphicspath{{./fig/}} % папка с изображениями
%%\usepackage[tikz]{bclogo}
%%\usepackage{wrapfig}
%%\usepackage{tikz}
%%\usetikzlibrary{shapes.geometric,backgrounds,calc,positioning,arrows,
%%                chains,decorations.markings,patterns,fadings,shapes.multipart,trees}
%
%%\usepackage{setspace} % для полуторного интервала
%%\onehalfspacing % сам полуторный интервал
%
\usepackage{indentfirst} % отступ в первом абзаце
%
%% точки в секция и подсекциях
%% \usepackage{secdot}
%% \sectiondot{subsection}

% Подавление висячих строк
\clubpenalty=10000
\widowpenalty=10000

%% колонтитулы
%%\usepackage{fancybox,fancyhdr}
%%\pagestyle{fancy}
%%\fancyhf{}
%%\fancyhead[C]{\small{C++ and Qt}} % шапка верхнего колонтитула!!!
%%\fancyfoot[C]{\small{\thepage}} % шапка нижнего колонтитула!!!
%
%% стили листингов кодов

\definecolor{commentGreen}{rgb}{0,0.6,0}
\lstdefinestyle{CPlusPlus}{
  language=C++,
  basicstyle=\small\ttfamily,
  breakatwhitespace=true,
  breaklines=true,
  showstringspaces=false,
  keywordstyle=\color{blue}\ttfamily,
  stringstyle=\color{red}\ttfamily,
  commentstyle=\color{commentGreen}\ttfamily,
  morecomment=[l][\color{magenta}]{\#},
  numbers=left,
  xleftmargin=1cm
}
\lstset{style=CPlusPlus}
\lstset{inputencoding=utf8, extendedchars=\true}
\lstset{escapeinside={(*@}{@*)}}

\addto\captionsrussian{\renewcommand\chaptername{Раздел}}
\newcommand{\chaptershort}[1]{
   \markboth{\MakeUppercase{\chaptername{ }\thechapter.\hspace{1em}#1}}{}
}
